%-----------------------------------------------------------------------------------------------------------------------------------------------%
%	The MIT License (MIT)
%
%	Copyright (c) 2019 Jan Küster
%
%	Permission is hereby granted, free of charge, to any person obtaining a copy
%	of this software and associated documentation files (the "Software"), to deal
%	in the Software without restriction, including without limitation the rights
%	to use, copy, modify, merge, publish, distribute, sublicense, and/or sell
%	copies of the Software, and to permit persons to whom the Software is
%	furnished to do so, subject to the following conditions:
%	
%	THE SOFTWARE IS PROVIDED "AS IS", WITHOUT WARRANTY OF ANY KIND, EXPRESS OR
%	IMPLIED, INCLUDING BUT NOT LIMITED TO THE WARRANTIES OF MERCHANTABILITY,
%	FITNESS FOR A PARTICULAR PURPOSE AND NONINFRINGEMENT. IN NO EVENT SHALL THE
%	AUTHORS OR COPYRIGHT HOLDERS BE LIABLE FOR ANY CLAIM, DAMAGES OR OTHER
%	LIABILITY, WHETHER IN AN ACTION OF CONTRACT, TORT OR OTHERWISE, ARISING FROM,
%	OUT OF OR IN CONNECTION WITH THE SOFTWARE OR THE USE OR OTHER DEALINGS IN
%	THE SOFTWARE.
%	
%
%-----------------------------------------------------------------------------------------------------------------------------------------------%


%============================================================================%
%
%	DOCUMENT DEFINITION
%
%============================================================================%

%we use article class because we want to fully customize the page and don't use a cv template
\documentclass[10pt,A4]{article}	


%----------------------------------------------------------------------------------------
%	ENCODING
%----------------------------------------------------------------------------------------

% we use utf8 since we want to build from any machine
\usepackage[utf8]{inputenc}		

%----------------------------------------------------------------------------------------
%	LOGIC
%----------------------------------------------------------------------------------------

% provides \isempty test
\usepackage{xstring, xifthen}

%----------------------------------------------------------------------------------------
%	FONT BASICS
%----------------------------------------------------------------------------------------

% some tex-live fonts - choose your own

%\usepackage[defaultsans]{droidsans}
%\usepackage[default]{comfortaa}
%\usepackage{cmbright}
\usepackage[default]{raleway}
%\usepackage{fetamont}
%\usepackage[default]{gillius}
%\usepackage[light,math]{iwona}
%\usepackage[thin]{roboto} 

% set font default
\renewcommand*\familydefault{\sfdefault} 	
\usepackage[T1]{fontenc}

% more font size definitions
\usepackage{moresize}

%----------------------------------------------------------------------------------------
%	FONT AWESOME ICONS
%---------------------------------------------------------------------------------------- 

% include the fontawesome icon set
\usepackage{fontawesome}

% use to vertically center content
% credits to: http://tex.stackexchange.com/questions/7219/how-to-vertically-center-two-images-next-to-each-other
\newcommand{\vcenteredinclude}[1]{\begingroup
\setbox0=\hbox{\includegraphics{#1}}%
\parbox{\wd0}{\box0}\endgroup}

% use to vertically center content
% credits to: http://tex.stackexchange.com/questions/7219/how-to-vertically-center-two-images-next-to-each-other
\newcommand*{\vcenteredhbox}[1]{\begingroup
\setbox0=\hbox{#1}\parbox{\wd0}{\box0}\endgroup}

% icon shortcut
\newcommand{\icon}[3] { 							
	\makebox(#2, #2){\textcolor{maincol}{\csname fa#1\endcsname}}
}	

% icon with text shortcut
\newcommand{\icontext}[4]{ 						
	\vcenteredhbox{\icon{#1}{#2}{#3}}  \hspace{2pt}  \parbox{0.9\mpwidth}{\textcolor{#4}{#3}}
}

% icon with website url
\newcommand{\iconhref}[5]{ 						
    \vcenteredhbox{\icon{#1}{#2}{#5}}  \hspace{2pt} \href{#4}{\textcolor{#5}{#3}}
}

% icon with email link
\newcommand{\iconemail}[5]{ 						
    \vcenteredhbox{\icon{#1}{#2}{#5}}  \hspace{2pt} \href{mailto:#4}{\textcolor{#5}{#3}}
}

%----------------------------------------------------------------------------------------
%	PAGE LAYOUT  DEFINITIONS
%----------------------------------------------------------------------------------------

% page outer frames (debug-only)
% \usepackage{showframe}		

% we use paracol to display breakable two columns
\usepackage{paracol}

% define page styles using geometry
\usepackage[a4paper]{geometry}

% remove all possible margins
\geometry{top=1cm, bottom=1cm, left=1cm, right=1cm}

\usepackage{fancyhdr}
\pagestyle{empty}

% space between header and content
% \setlength{\headheight}{0pt}

% indentation is zero
\setlength{\parindent}{0mm}

%----------------------------------------------------------------------------------------
%	TABLE /ARRAY DEFINITIONS
%---------------------------------------------------------------------------------------- 

% extended aligning of tabular cells
\usepackage{array}

% custom column right-align with fixed width
% use like p{size} but via x{size}
\newcolumntype{x}[1]{%
>{\raggedleft\hspace{0pt}}p{#1}}%


%----------------------------------------------------------------------------------------
%	GRAPHICS DEFINITIONS
%---------------------------------------------------------------------------------------- 

%for header image
\usepackage{graphicx}

% use this for floating figures
% \usepackage{wrapfig}
% \usepackage{float}
% \floatstyle{boxed} 
% \restylefloat{figure}

%for drawing graphics		
\usepackage{tikz}				
\usetikzlibrary{shapes, backgrounds,mindmap, trees}

%----------------------------------------------------------------------------------------
%	Color DEFINITIONS
%---------------------------------------------------------------------------------------- 
\usepackage{transparent}
\usepackage{color}

% primary color
\definecolor{maincol}{RGB}{ 225, 0, 0 }

% accent color, secondary
% \definecolor{accentcol}{RGB}{ 250, 150, 10 }

% dark color
\definecolor{darkcol}{RGB}{ 70, 70, 70 }

% light color
\definecolor{lightcol}{RGB}{245,245,245}


% Package for links, must be the last package used
\usepackage[hidelinks]{hyperref}

% returns minipage width minus two times \fboxsep
% to keep padding included in width calculations
% can also be used for other boxes / environments
\newcommand{\mpwidth}{\linewidth-\fboxsep-\fboxsep}
	


%============================================================================%
%
%	CV COMMANDS
%
%============================================================================%

%----------------------------------------------------------------------------------------
%	 CV LIST
%----------------------------------------------------------------------------------------

% renders a standard latex list but abstracts away the environment definition (begin/end)
\newcommand{\cvlist}[1] {
	\begin{itemize}{#1}\end{itemize}
}

%----------------------------------------------------------------------------------------
%	 CV TEXT
%----------------------------------------------------------------------------------------

% base class to wrap any text based stuff here. Renders like a paragraph.
% Allows complex commands to be passed, too.
% param 1: *any
\newcommand{\cvtext}[1] {
	\begin{tabular*}{1\mpwidth}{p{0.98\mpwidth}}
		\parbox{1\mpwidth}{#1}
	\end{tabular*}
}

%----------------------------------------------------------------------------------------
%	CV SECTION
%----------------------------------------------------------------------------------------

% Renders a a CV section headline with a nice underline in main color.
% param 1: section title
\newcommand{\cvsection}[1] {
	\vspace{14pt}
	\cvtext{
		\textbf{\LARGE{\textcolor{darkcol}{\uppercase{#1}}}}\\[-4pt]
		\textcolor{maincol}{ \rule{0.1\textwidth}{2pt} } \\
	}
}

%----------------------------------------------------------------------------------------
%	META SKILL
%----------------------------------------------------------------------------------------

% Renders a progress-bar to indicate a certain skill in percent.
% param 1: name of the skill / tech / etc.
% param 2: level (for example in years)
% param 3: percent, values range from 0 to 1
\newcommand{\cvskill}[3] {
	\begin{tabular*}{1\mpwidth}{p{0.72\mpwidth}  r}
 		\textcolor{black}{\textbf{#1}} & \textcolor{maincol}{#2}\\
	\end{tabular*}%
	
	\hspace{4pt}
	\begin{tikzpicture}[scale=1,rounded corners=2pt,very thin]
		\fill [lightcol] (0,0) rectangle (1\mpwidth, 0.15);
		\fill [maincol] (0,0) rectangle (#3\mpwidth, 0.15);
  	\end{tikzpicture}%
}


%----------------------------------------------------------------------------------------
%	 CV EVENT
%----------------------------------------------------------------------------------------

% Renders a table and a paragraph (cvtext) wrapped in a parbox (to ensure minimum content
% is glued together when a pagebreak appears).
% Additional Information can be passed in text or list form (or other environments).
% the work you did
% param 1: time-frame i.e. Sep 14 - Jan 15 etc.
% param 2:	 event name (job position etc.)
% param 3: Customer, Employer, Industry
% param 4: Short description
% param 5: work done (optional)
% param 6: technologies include (optional)
% param 7: achievements (optional)
\newcommand{\cvevent}[7] {
	
	% we wrap this part in a parbox, so title and description are not separated on a pagebreak
	% if you need more control on page breaks, remove the parbox
	\parbox{\mpwidth}{
		\begin{tabular*}{1\mpwidth}{p{0.72\mpwidth}  r}
	 		\textcolor{black}{\textbf{#2}} & \colorbox{maincol}{\makebox[0.25\mpwidth]{\textcolor{white}{#1}}} \\
			\textcolor{maincol}{\textbf{#3}} & \\
		\end{tabular*}\\[8pt]
	
		\ifthenelse{\isempty{#4}}{}{
			\cvtext{#4}\\
		}
	}

	\ifthenelse{\isempty{#5}}{}{
		\vspace{9pt}
		{#5}
	}

	\ifthenelse{\isempty{#6}}{}{
		\vspace{9pt}
		\cvtext{\textbf{Technologies include:}}\\
		{#6}
	}

	\ifthenelse{\isempty{#7}}{}{
		\vspace{9pt}
		\cvtext{\textbf{Achievements include:}}\\
		{#7}
	}
	\vspace{14pt}
}

%----------------------------------------------------------------------------------------
%	 CV META EVENT
%----------------------------------------------------------------------------------------

% Renders a CV event on the sidebar
% param 1: title
% param 2: subtitle (optional)
% param 3: customer, employer, etc,. (optional)
% param 4: info text (optional)
\newcommand{\cvmetaevent}[4] {
	\textcolor{maincol} {\cvtext{\textbf{\begin{flushleft}#1\end{flushleft}}}}

	\ifthenelse{\isempty{#2}}{}{
	\textcolor{darkcol} {\cvtext{\textbf{#2}} }
	}

	\ifthenelse{\isempty{#3}}{}{
		\cvtext{{ \textcolor{darkcol} {#3} }}\\
	}

	\cvtext{#4}\\[14pt]
}

%---------------------------------------------------------------------------------------
%	 CODE
%----------------------------------------------------------------------------------------

% Renders a qrcode image (centered, relative to the parentwidth)
% param 1: percent width, from 0 to 1
\newcommand{\cvqrcode}[1] {
	\begin{center}
		\includegraphics[width={#1}\mpwidth]{qrcode}
	\end{center}
}


%============================================================================%
%
%
%
%	DOCUMENT CONTENT
%
%
%
%============================================================================%
\begin{document}
\columnratio{0.31}
\setlength{\columnsep}{2.2em}
\setlength{\columnseprule}{4pt}
\colseprulecolor{lightcol}
\begin{paracol}{2}
\begin{leftcolumn}
%---------------------------------------------------------------------------------------
%	META IMAGE
%----------------------------------------------------------------------------------------
\includegraphics[width=\linewidth]{xp.JPG}	%trimming relative to image size

%---------------------------------------------------------------------------------------
%	META SKILLS
%----------------------------------------------------------------------------------------
\cvsection{SKILLS}

\cvskill{Java} {10+ yrs} {1} \\[-2pt]

\cvskill{Automation} {10+ yrs} {1} \\[-2pt]

\cvskill{AWS} {9+ yrs} {1} \\[-2pt]

\cvskill{Agile \& DevOps} {9+ yrs} {1} \\[-2pt]

\cvskill{Infrastructure as Code} {7+ yrs} {1} \\[-2pt]

\cvskill{Jenkins} {7+ yrs} {1} \\[-2pt]

\cvskill{Python} {5+ yrs} {0.7} \\[-2pt]

\cvskill{Chaos Engineering} {4+ yrs} {0.7} \\[-2pt]

\cvskill{k8s, Helm} {4+ yrs} {0.7} \\[-2pt]

\cvskill{Observability} {1+ yrs} {0.5} \\[-2pt]





\vfill\null
\cvsection{CONTACT}
	
\icontext{MapMarker}{12}{Stuttgart, Germany}{black}\\[6pt]
\icontext{MobilePhone}{12}{+49 175 9503 262}{black}\\[6pt]
\iconemail{Envelope}{12}{liuxp87@gmail.com}{liuxp87@gmail.com}{black}\\[6pt]



%---------------------------------------------------------------------------------------
%	EDUCATION
%----------------------------------------------------------------------------------------
\newpage
\cvsection{EDUCATION}

\cvmetaevent
{2011 - 2014}
{Master of Science}
{RWTH Aachen University\\ Aachen, Germany}
{Software System Engineering}

\cvmetaevent
{2005 - 2009}
{Bachelor of Engineering}
{Southeast University\\Nanjing, China}
{Software Engineering}

\cvsection{LANGUAGE}

\cvskill{Chinese} {Native} {1} \\[-2pt]

\cvskill{English} {C1} {0.9} \\[-2pt]

\cvskill{German} {A2} {0.4} \\[-2pt]





\cvsection{WORKS}

\cvmetaevent
{Chaos Engineering}
{chaostoolkit-saltstack}
{https://github.com/chaostoolkit-incubator/chaostoolkit-saltstack}
{Owner of \\Open-Sourced Project}

\cvmetaevent
{Chaos Engineering}
{chaostoolkit-azure}
{https://github.com/chaostoolkit-incubator/chaostoolkit-azure}
{Contributer of\\ Open-Sourced Project}



\vfill\null
\cvqrcode{0.6}



\end{leftcolumn}
\begin{rightcolumn}
%---------------------------------------------------------------------------------------
%	TITLE  HEADER
%----------------------------------------------------------------------------------------
\fcolorbox{white}{darkcol}{\begin{minipage}[c][3.5cm][c]{1\mpwidth}
	\begin {center}
		\HUGE{ \textbf{ \textcolor{white}{ \uppercase{ XIAOPENG LIU } } } } \\[-24pt]
		\textcolor{white}{ \rule{0.1\textwidth}{1.25pt} } \\[4pt]
		\large{ \textcolor{white} {Cloud DevSecOps Architect} }
	\end {center}
\end{minipage}} \\[14pt]
\vspace{-12pt}

%---------------------------------------------------------------------------------------
%	PROFILE
%----------------------------------------------------------------------------------------
\vfill\null
\cvsection{PROFILE}

\cvtext{
	- Guru and Advocator for Chaos Engineering. \\
	- Test Automation Expert with over a decade of experience. \\
	- Dedicated DevSecOps for nearly 10 years on both software and cloud-native infrastructures.
	
}

%---------------------------------------------------------------------------------------
%	WORK EXPERIENCE
%----------------------------------------------------------------------------------------
\vfill\null
\cvsection{WORK EXPERIENCE}

\cvevent
	{08/2023 - NOW}
	{Cloud DevSecOps Architect}
	{Mercedes-Benz AG}
	{Focus on IT system observability via Datadog, by providing multiple integration solutions through HELIOS platform. Supporting the sales and aftersale systems to setup SRE practices.
		Supporting HELIOS DevOps eco-system, improve the DevOps efficiencies and stability.}
	{\cvlist{
		\item Enable various integration to Datadog
		\item Advocate SRE practices, and improve operation tools
		\item Support CI/CD tools, improve the capacity and stability
		\item Practice Chaos Engineering
	}}
	{}
	{\cvlist{
		\item Established Datadog Integration with Kong, ArgoCD, AWS RDS, Kyverno etc.
		\item Maintain and improved Datadog agent installation in the platform
		\item Established Dashboard as Code solution to support business health KPI measurement
		\item Purple Belt of MB Security program
	}}

\vfill\null
\cvevent
	{11/2018 - 08/2023}
	{Solution Architect}
	{Mercedes-Benz Greater China}
	{Provided cloud-native infrastructure architecture and solutions for the business application team. Applied Agile and DevSecOps practices, leading a DevOps team to develop a platform that shorten infrastructure delivery time from two weeks to half a day via InfraAsCode. Developed a suite of operational tools for the enterprise technical platform and created a Chaos Engineering SaaS solution. Collaborated with enterprise architects to define and promote DevSecOps principles and standards.}
	{\cvlist{
		\item Establish and implement infrastructure as code solution in AWS CN
		\item Defined MB IT China infrastructure standards
		\item OneAPI owner in MB IT China
		\item Establish Chaos Engineering service in MB IT China
		\item FOSS and security topics
	}}
	{}
	{\cvlist{
		\item Shorten the AWS infrastructure delivery from 2 weeks to half day
		\item Successfully lead OneAPI migration from EdgeMicro to Kong
		\item Successfully lead pilot of DevOps SNOW application in MB IT China
		\item Support various FOSS and security topics
	}}

\vfill\null
\cvevent
	{09/2014 - 11/2018}
	{QA Team Lead / Principal Quality Engineer / Senior Quality Engineer}
	{Thomson Reuters}
	{Lead an automation test team to support various business projects. Maintain and develop new test frameworks and tools. Investigation and implementation on AWS migration solutions. Establishing software pipelines for DevOps approach to support both on-premises release and Agile approaches.
	}
	{\cvlist{
		\item Automation framework development and test management
		\item Lead team to cutting-edge techniques
		\item Embrace cloud migrations
		\item Practice cloud-native and DevOps
	}}
	{}
	{\cvlist{
		\item  Improved UFT/QTP automation framework, shorten the execution time from 1 week to 2 days
		\item  Discovered various critical issues in the commodity product
		\item  Lead the pilot AWS migration of CI/CD pipeline and database transactions.
	}}

\vfill\null
\cvevent
	{03/2012 - 12/2013}
	{Assistant Researcher}
	{i3@RWTH Aachen University}
	{Working on website fast feedback system in JavaScript including test with MBD and generative software engineering}
	{}
	{}
	{}

\vfill\null
\cvevent
	{09/2009 - 08/2011}
	{QA Engineer}
	{Advent Software}
	{Work on automation testing in Silktest and automation test framework training}
	{}{}{}


% hotfixes to create fake-space to ensure the whole height is used
\mbox{}
\vfill
\mbox{}
\vfill
\mbox{}
\vfill
\mbox{}
\end{rightcolumn}
\end{paracol}
\end{document}

